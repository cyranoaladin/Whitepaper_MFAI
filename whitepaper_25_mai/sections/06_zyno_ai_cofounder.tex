\section{Zyno — The AI Co-Founder™}

\begin{mfai-box}{Strategic Role of Zyno}{robot}
Zyno is not a chatbot. It is an AI-native co-founder, embedded into the MFAI protocol to guide builders across every phase of activation — from learning to launching, from ideation to governance.
\end{mfai-box}

\vspace{1em}

\subsection*{Why an AI Co-Founder?}

In the post-Web2 economy, learning is continuous, and decisions are complex. Builders need more than access to tools — they need strategic collaboration. Zyno operates as a dynamic, context-aware AI agent that helps you:

\begin{itemize}
  \item \mfaiproof{Understand the protocol} — break down mechanisms like Skillchain Mining™ and Governance.
  \item \mfaiproof{Plan your path} — suggest learning modules, projects, and launch strategies.
  \item \mfaiproof{Co-write with precision} — prompt engineering, documentation, tokenomics modeling, etc.
  \item \mfaiproof{Generate proof} — validate actions, issue credentials, or prepare submissions to incubators or governance.
\end{itemize}

\vspace{1.5em}

\subsection*{Zyno’s Dual Intelligence Engine}

\begin{mfai-note}
Zyno is powered by two proprietary modules: \textbf{AEPO} and \textbf{AECO}, which together form its adaptive intelligence core.
\end{mfai-note}

\begin{itemize}
  \item \mfaiproof{AEPO — Ask Engine Prompt Optimization:} Refines user queries using project state, goals, and prior context to produce more actionable, targeted prompts.
  \item \mfaiproof{AECO — Answer Engine Clarity Optimization:} Synthesizes responses via knowledge graphs and RAG (retrieval augmented generation) tuned to MFAI’s internal knowledge base.
\end{itemize}

\vspace{2em}

\subsection*{Zyno’s Modes of Operation}

\begin{center}
\begin{tikzpicture}[
  node distance=1.3cm and 1.8cm,
  mode/.style={
    draw=solana-green!70!black,
    fill=solana-green!10,
    font=\bfseries\small,
    text centered,
    text width=5.4cm,
    rounded corners=3pt,
    minimum height=1.2cm
  }
]

\node[mode] (mentor) {Mentor Mode\\\scriptsize{Personalized learning and career planning}};
\node[mode, below=of mentor] (builder) {Builder Mode\\\scriptsize{Ideation, tokenomics, launch planning}};
\node[mode, below=of builder] (analyst) {Analyst Mode\\\scriptsize{Governance support, proposal structuring}};
\node[mode, below=of analyst] (assistant) {Assistant Mode\\\scriptsize{Day-to-day productivity and prompt crafting}};

\end{tikzpicture}
\end{center}

\vspace{1em}

\begin{mfai-box}{Zyno Stack™ Overview}{microchip}
\textbf{Zyno Stack™} = AEPO + AECO + MFAI Graph + Prompt Layer + Access Control.
\end{mfai-box}

\vspace{2em}

\begin{mfai-warning}
Zyno is an \textit{AI-native actor}, not just a user interface. Every interaction is logged, contextualized, and convertible into on-chain actions or proofs. This makes Zyno a \mfaiword{trust layer} for cognitive collaboration.
\end{mfai-warning}
